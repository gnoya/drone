%%%%%%%%%%%%%%%%%%%%%%%%%%%%%%%%%%%%%%%%%%%%%%%%%%%%%%%%%%%%%%%%%%%%%%%%%%%%%%%%
%2345678901234567890123456789012345678901234567890123456789012345678901234567890
%        1         2         3         4         5         6         7         8

\documentclass[letterpaper, 10 pt, conference]{ieeeconf}  % Comment this line out if you need a4paper

%\documentclass[a4paper, 10pt, conference]{ieeeconf}      % Use this line for a4 paper

\IEEEoverridecommandlockouts                              % This command is only needed if 
                                                          % you want to use the \thanks command

\overrideIEEEmargins                                      % Needed to meet printer requirements.

%In case you encounter the following error:
%Error 1010 The PDF file may be corrupt (unable to open PDF file) OR
%Error 1000 An error occurred while parsing a contents stream. Unable to analyze the PDF file.
%This is a known problem with pdfLaTeX conversion filter. The file cannot be opened with acrobat reader
%Please use one of the alternatives below to circumvent this error by uncommenting one or the other
%\pdfobjcompresslevel=0
%\pdfminorversion=4

% See the \addtolength command later in the file to balance the column lengths
% on the last page of the document

% The following packages can be found on http:\\www.ctan.org
%\usepackage{graphics} % for pdf, bitmapped graphics files
\usepackage{graphicx}
\usepackage{amsmath}
\usepackage{amssymb}
\usepackage{url}
\usepackage{subfigure}
\usepackage{booktabs}
\graphicspath{{./Figures/}}

\usepackage{cite}  % converts [1],[2] to [1,2]
\makeatletter
\def\@citex[#1]#2{\leavevmode
\let\@citea\@empty
\@cite{\@for\@citeb:=#2\do
{\@citea\def\@citea{,\penalty\@m\ }%
\edef\@citeb{\expandafter\@firstofone\@citeb\@empty}%
\if@filesw\immediate\write\@auxout{\string\citation{\@citeb}}\fi
\@ifundefined{b@\@citeb}{\hbox{\reset@font\bfseries ?}%
\G@refundefinedtrue
\@latex@warning
{Citation `\@citeb' on page \thepage \space undefined}}%
{\@cite@ofmt{\csname b@\@citeb\endcsname}}}}{#1}}
\makeatother


%\usepackage{dblfloatfix}
%\usepackage{epsfig} % for postscript graphics files
%\usepackage{mathptmx} % assumes new font selection scheme installed
%\usepackage{times} % assumes new font selection scheme installed
%\usepackage{amsmath} % assumes amsmath package installed
%\usepackage{amssymb}  % assumes amsmath package installed

\title{\LARGE \bf
Lectura y Monitoreo Inalámbrico de Sensores y Controladores de un Cuadricóptero utilizando ROS
}


\author{Gabriel Noya Doval$^{1}$, Carlos Serrano Barreto$^{1}$% <-this % stops a space
%\thanks{*This work was not supported by any organization}% <-this % stops a space
\thanks{$^{1}$Grupo de Investigación de Mecatrónica, Universidad Simón Bolívar, Caracas, Venezuela.
        {\tt\small \{13-10982, 13-00000\}@usb.ve}}%
}


\begin{document}

\maketitle
\thispagestyle{empty}
\pagestyle{empty}
%%%%%%%%%%%%%%%%%%%%%%%%%%%%%%%%%%%%%%%%%%%%%%%%%%%%%%%%%%%%%%%%%%%%%%%%%%%%%%%%
\begin{abstract}

Resumen hehexD
\\
\end{abstract}

\begin{keywords}
Cuadricóptero, controladores, sensores, ROS
\end{keywords}
%%%%%%%%%%%%%%%%%%%%%%%%%%%%%%%%%%%%%%%%%%%%%%%%%%%%%%%%%%%%%%%%%%%%%%%%%%%%%%%%

\section{DESCRIPCIÓN DEL PROYECTO}

Bla bla bla bla Bla bla bla bla Bla bla bla bla Bla bla bla bla
Bla bla bla bla Bla bla bla bla Bla bla bla bla Bla bla bla bla
Bla bla bla bla Bla bla bla bla Bla bla bla bla Bla bla bla bla
Bla bla bla bla Bla bla bla bla Bla bla bla bla Bla bla bla bla
Bla bla bla bla Bla bla bla bla Bla bla bla bla Bla bla bla bla
Bla bla bla bla Bla bla bla bla Bla bla bla bla Bla bla bla bla
%\begin{figure}[htpb]
%	\centering
%	\includegraphics[width=0.99\columnwidth]{histogram}
%	\caption{Histogram of pixels inside the roundabout bounding box in Figure~\ref{stereo_results_example_2} versus the distance they represent. In this example, the distance of the traffic sign to the vehicle is 11.6 meters.}
%	\label{disparity_histogram}
%\end{figure}

\section{DESCRIPCIÓN DEL CUADRICÓPTERO}
Bla bla bla bla Bla bla bla bla Bla bla bla bla Bla bla bla bla
Bla bla bla bla Bla bla bla bla Bla bla bla bla Bla bla bla bla
Bla bla bla bla Bla bla bla bla Bla bla bla bla Bla bla bla bla
Bla bla bla bla Bla bla bla bla Bla bla bla bla Bla bla bla bla
Bla bla bla bla Bla bla bla bla Bla bla bla bla Bla bla bla bla
Bla bla bla bla Bla bla bla bla Bla bla bla bla Bla bla bla bla
Bla bla bla bla Bla bla bla bla Bla bla bla bla Bla bla bla bla
Bla bla bla bla Bla bla bla bla Bla bla bla bla Bla bla bla bla
Bla bla bla bla Bla bla bla bla Bla bla bla bla Bla bla bla bla
Bla bla bla bla Bla bla bla bla Bla bla bla bla Bla bla bla bla
Bla bla bla bla Bla bla bla bla Bla bla bla bla Bla bla bla bla
Bla bla bla bla Bla bla bla bla Bla bla bla bla Bla bla bla bla
Bla bla bla bla Bla bla bla bla Bla bla bla bla Bla bla bla bla
Bla bla bla bla Bla bla bla bla Bla bla bla bla Bla bla bla bla
Bla bla bla bla Bla bla bla bla Bla bla bla bla Bla bla bla bla
Bla bla bla bla Bla bla bla bla Bla bla bla bla Bla bla bla bla
Bla bla bla bla Bla bla bla bla Bla bla bla bla Bla bla bla bla
Bla bla bla bla Bla bla bla bla Bla bla bla bla Bla bla bla bla
Bla bla bla bla Bla bla bla bla Bla bla bla bla Bla bla bla bla
Bla bla bla bla Bla bla bla bla Bla bla bla bla Bla bla bla bla
Bla bla bla bla Bla bla bla bla Bla bla bla bla Bla bla bla bla
\section{PROCEDIMIENTO}
Bla bla bla bla Bla bla bla bla Bla bla bla bla Bla bla bla bla
Bla bla bla bla Bla bla bla bla Bla bla bla bla Bla bla bla bla
Bla bla bla bla Bla bla bla bla Bla bla bla bla Bla bla bla bla
Bla bla bla bla Bla bla bla bla Bla bla bla bla Bla bla bla bla
Bla bla bla bla Bla bla bla bla Bla bla bla bla Bla bla bla bla
Bla bla bla bla Bla bla bla bla Bla bla bla bla Bla bla bla bla
Bla bla bla bla Bla bla bla bla Bla bla bla bla Bla bla bla bla
Bla bla bla bla Bla bla bla bla Bla bla bla bla Bla bla bla bla
Bla bla bla bla Bla bla bla bla Bla bla bla bla Bla bla bla bla
Bla bla bla bla Bla bla bla bla Bla bla bla bla Bla bla bla bla
Bla bla bla bla Bla bla bla bla Bla bla bla bla Bla bla bla bla
Bla bla bla bla Bla bla bla bla Bla bla bla bla Bla bla bla bla
Bla bla bla bla Bla bla bla bla Bla bla bla bla Bla bla bla bla
Bla bla bla bla Bla bla bla bla Bla bla bla bla Bla bla bla bla
Bla bla bla bla Bla bla bla bla Bla bla bla bla Bla bla bla bla
\subsection{DIAGRAMA DE BLOQUES}
Bla bla bla bla Bla bla bla bla Bla bla bla bla Bla bla bla bla
Bla bla bla bla Bla bla bla bla Bla bla bla bla Bla bla bla bla
Bla bla bla bla Bla bla bla bla Bla bla bla bla Bla bla bla bla
Bla bla bla bla Bla bla bla bla Bla bla bla bla Bla bla bla bla
Bla bla bla bla Bla bla bla bla Bla bla bla bla Bla bla bla bla
Bla bla bla bla Bla bla bla bla Bla bla bla bla Bla bla bla bla
\subsection{PAQUETES A UTILIZAR}
Bla bla bla bla Bla bla bla bla Bla bla bla bla Bla bla bla bla
Bla bla bla bla Bla bla bla bla Bla bla bla bla Bla bla bla bla
Bla bla bla bla Bla bla bla bla Bla bla bla bla Bla bla bla bla
Bla bla bla bla Bla bla bla bla Bla bla bla bla Bla bla bla bla
Bla bla bla bla Bla bla bla bla Bla bla bla bla Bla bla bla bla
Bla bla bla bla Bla bla bla bla Bla bla bla bla Bla bla bla bla


\section{CRONOGRAMA DE ACTIVIDADES}
Bla bla bla bla Bla bla bla bla Bla bla bla bla Bla bla bla bla
Bla bla bla bla Bla bla bla bla Bla bla bla bla Bla bla bla bla
Bla bla bla bla Bla bla bla bla Bla bla bla bla Bla bla bla bla
Bla bla bla bla Bla bla bla bla Bla bla bla bla Bla bla bla bla
Bla bla bla bla Bla bla bla bla Bla bla bla bla Bla bla bla bla
Bla bla bla bla Bla bla bla bla Bla bla bla bla Bla bla bla bla


%\addtolength{\textheight}{-7cm}   % This command serves to balance the column lengths
                                  % on the last page of the document manually. It shortens
                                  % the textheight of the last page by a suitable amount.
                                  % This command does not take effect until the next page
                                  % so it should come on the page before the last. Make
                                  % sure that you do not shorten the textheight too much.

%%%%%%%%%%%%%%%%%%%%%%%%%%%%%%%%%%%%%%%%%%%%%%%%%%%%%%%%%%%%%%%%%%%%%%%%%%%%%%%%
%\section*{APPENDIX}
%

%%%%%%%%%%%%%%%%%%%%%%%%%%%%%%%%%%%%%%%%%%%%%%%%%%%%%%%%%%%%%%%%%%%%%%%%%%%%%%%%

%\begin{thebibliography}{99}
%\item 
%\end{thebibliography}


\bibliography{references}
\bibliographystyle{ieeetr}

\end{document}
